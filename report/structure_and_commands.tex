\documentclass[oneside,openright,a4paper, 11pt]{report}
\usepackage[resetfonts]{cmap}
\usepackage[T1]{fontenc}
\usepackage[french]{babel}
\usepackage[utf8]{inputenc}
\usepackage{float}
\usepackage{subfig}
\usepackage{fancyhdr}
\usepackage{lastpage}
\usepackage[nottoc]{tocbibind}
\usepackage[ruled,vlined]{algorithm2e}

% Needed to use biblatex instead of bibtex
%\usepackage[style=authoryear-ibid,backend=biber,sorting=anyt]{biblatex}
%\addbibresource{biblio.bib}

% Image management
\usepackage[pdftex]{graphicx}
% Margins
\usepackage{geometry}
\geometry{
	paper=a4paper,
	total={160mm,257mm},
	left=25mm,
	top=10mm,
	includeheadfoot}
% Paragraph indentation
\setlength{\parindent}{0em}
% Link format
\usepackage[colorlinks = true, linkcolor = blue, urlcolor  = blue, citecolor = blue, anchorcolor = blue]{hyperref}
% Caption options
\usepackage[format=hang, font=footnotesize, justification=centerlast]{caption}

% TODO notes
% Usage:
% Note in the margin: \todo{this is a todo note}
% Inline note: \todo[inline]{this is a todo note}
% To disable the display of todo notes: %\usepackage[disable]{todonotes}
\usepackage{todonotes}
% Citations quotes
\usepackage[autostyle=true]{csquotes}
% Footnotes fixed at the bottom of the page
\usepackage[bottom]{footmisc}

\usepackage{listings}
\usepackage{xcolor}
\usepackage{colortbl}
\definecolor{darkgreen}{rgb}{0,0.6,0}
\lstset
{
	basicstyle=\footnotesize,
	breakatwhitespace=false,
	breaklines=true,
	commentstyle=\color{darkgreen}\bfseries,
	keepspaces=true,
	keywordstyle=\color{blue}\bfseries,
	language=C++,
	showspaces=false,
	showstringspaces=false,
	showtabs=false,
	stringstyle=\color{red}\bfseries,
	tabsize=2
}

\usepackage{lipsum}
\usepackage[framemethod=tikz]{mdframed}

% English words
\newcommand{\eng}[1]{
	\kern-1ex \textit{#1}~\kern-1ex
}

%%%%%%%%%%%%%%%%%%%%%%%%%%%%%%%%%%%%%%%%%
% Lachaise Assignment
% Structure Specification File
% Version 1.0 (26/6/2018)
%
% This template originates from:
% http://www.LaTeXTemplates.com
%
% Authors:
% Marion Lachaise & François Févotte
% Vel (vel@LaTeXTemplates.com)
%
% License:
% CC BY-NC-SA 3.0 (http://creativecommons.org/licenses/by-nc-sa/3.0/)
% 
%%%%%%%%%%%%%%%%%%%%%%%%%%%%%%%%%%%%%%%%%
%----------------------------------------------------------------------------------------
%	COMMAND LINE ENVIRONMENT
%----------------------------------------------------------------------------------------

% Usage:
% \begin{commandline}
%	\begin{verbatim}
%		$ ls
%		
%		Applications	Desktop	...
%	\end{verbatim}
% \end{commandline}
\mdfdefinestyle{commandline}{
	leftmargin=10pt,
	rightmargin=10pt,
	innerleftmargin=15pt,
	middlelinecolor=black!50!white,
	middlelinewidth=2pt,
	frametitlerule=false,
	backgroundcolor=black!5!white,
	frametitle={Command Line},
	frametitlefont={\normalfont\sffamily\color{white}\hspace{-1em}},
	frametitlebackgroundcolor=black!50!white,
	nobreak,
}
% Define a custom environment for command-line snapshots
\newenvironment{commandline}{
	\medskip
	\begin{mdframed}[style=commandline]
		}{
	\end{mdframed}
	\medskip
}

% Usage:
% \begin{warn}[optional title, defaults to "Warning:"]
%	Contents
% \end{warn}
\mdfdefinestyle{warning}{
	topline=false, bottomline=false,
	leftline=false, rightline=false,
	nobreak,
	singleextra={%
			\draw(P-|O)++(-0.5em,0)node(tmp1){};
			\draw(P-|O)++(0.5em,0)node(tmp2){};
			\fill[black,rotate around={45:(P-|O)}](tmp1)rectangle(tmp2);
			\node at(P-|O){\color{white}\scriptsize\bf !};
			\draw[very thick](P-|O)++(0,-1em)--(O);%--(O-|P);
		}
}

% Define a custom environment for warning text
\newenvironment{warn}[1][Warning:]{ % Set the default warning to "Warning:"
	\medskip
	\begin{mdframed}[style=warning]
		\noindent{\textbf{#1}}
		}{
	\end{mdframed}
}

% Usage:
% \begin{info}[optional title, defaults to "Info:"]
% 	contents
% 	\end{info}
\mdfdefinestyle{info}{%
topline=false, bottomline=false,
leftline=false, rightline=false,
nobreak,
singleextra={%
\fill[black](P-|O)circle[radius=0.4em];
\node at(P-|O){\color{white}\scriptsize\bf i};
\draw[very thick](P-|O)++(0,-0.8em)--(O);%--(O-|P);
}
}
% Define a custom environment for information
\newenvironment{info}[1][Info:]{ % Set the default title to "Info:"
	\medskip
	\begin{mdframed}[style=info]
		\noindent{\textbf{#1}}
		}{
	\end{mdframed}
}

